\begin{figure*}[htb]
\centering
\hfill
\begin{subfigure}{0.3\columnwidth}
        \includegraphics[width=\columnwidth]{./img/adni_subspace.pdf}
        \caption{Latent sub-space}
        %\label{fig:syn_log_likelihood}
\end{subfigure}
\hfill
\begin{subfigure}{0.5\columnwidth}
        \includegraphics[width=\columnwidth]{./img/adni_mri.pdf}
        \caption{MRI}
        %\label{fig:syn_log_likelihood}
\end{subfigure}
\hfill
\begin{subfigure}{0.5\columnwidth}
        \includegraphics[width=\columnwidth]{./img/adni_fdg.pdf}
        \caption{FDG}
        %\label{fig:syn_log_likelihood}
\end{subfigure}
\hfill
\begin{subfigure}{0.5\columnwidth}
        \includegraphics[width=\columnwidth]{./img/adni_amy.pdf}
        \caption{Amyloid}
        %\label{fig:syn_log_likelihood}
\end{subfigure}
\hfill
\caption{
Generation of imaging data from trajectories in the latent space.
(a) Normal aging trajectory ($Tr_1$) \textit{vs} Dementia aging trajectory ($Tr_2$) in the latent $2$D sub-space (\textit{cfr.} \figref{fig:stratification}).
Stars indicate the sampling points along trajectories.
The trajectories share the same origin.
MRIs (b), FDG (c), and Amyloid PET (d).
All the trajectories show a plausible evolution across disease and healthy conditions.
}
\label{fig:ageing}
\end{figure*}
