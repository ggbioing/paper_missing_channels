% The abstract should briefly summarize the contents of the paper in
% 150--250 words.
% \keywords{First keyword  \and Second keyword \and Another keyword.}
The joint modeling of different biomedical datasets requires the ability of consistently analyzing heterogeneous information in presence of often non-overlapping sets of views, or modalities (e.g. clinical scores, imaging data, or biological measurements), which may be missing for economical, ethical, or scarcity reasons.
This problem requires to deal with random and non-random missing data specific to each dataset, under the form of missing measurements corresponding to a given view.
%
We propose here a generative latent-variable model of the common variability across datasets, based on the estimation of a shared latent representation across views.
Our formulation allows to retrieve a consistent latent representation common to all data-types and datasets, even if datasets share only partially overlapping subsets of views.
%
Experiments on synthetic data show that our method is able to identify a common latent representation of multi-feature datasets, even when the overlap of views in the feature-sets across datasets is minimal.
% The resulting latent representation is generally equivalent to one obtained with a model trained with no missing data-types.
%
When analyzing independent dementia datasets containing heterogeneous clinical and imaging data modalities, our model is able to mitigate the absence of views in datasets.
Moreover, the common latent representation inferred with our model can be used to define robust classification models gathering the combined information across different datasets.
