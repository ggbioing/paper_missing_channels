% The abstract should briefly summarize the contents of the paper in
% 150--250 words.
% \keywords{First keyword  \and Second keyword \and Another keyword.}
The joint modeling of neuroimaging data across multiple datasets requires to consistently analyze high-dimensional and heterogeneous information in presence of often non-overlapping sets of views across data samples (e.g. imaging data, clinical scores, biological measurements).
This analysis is associated with the problem of missing information across datasets, which can take place in two forms:
missing at random (MAR), when the absence of a view is unpredictable and does not depend on the dataset (e.g. due to data corruption);
missing not at random (MNAR), when a specific view is absent by design for a specific dataset.
%
In order to take advantage of the increased variability and sample size when pooling together observations from many cohorts,
and at the same time cope with the ubiquitous problem of missing information,
we propose here a multi-task generative latent-variable model where the common variability across datasets stems from the estimation of a shared latent representation across views.
Our formulation allows to retrieve a consistent latent representation common to all views and datasets, even in the presence of missing information.
%
Simulations on synthetic data show that our method is able to identify a common latent representation of multi-view datasets, even when the compatibility across datasets is minimal.
%
When jointly analyzing multi-modal neuroimaging and clinical data from real independent dementia studies, our model is able to mitigate the absence of modalities without having to discard any available information.
Moreover, the common latent representation inferred with our model can be used to define robust classifiers gathering the combined information across different datasets.
%
To conclude, both on synthetic and real data experiments, our model compared favorably to state of the art benchmark methods, providing a more powerful exploitation of multi-modal observations with missing views.
%
Code is publicly available at \url{https://gitlab.inria.fr/epione\_ML/mcvae}.
