\section{Discussion}

In both the experiments on synthetic and real data, our model compared favorably with respect to state of the art benchmark methods.

An interesting result is the one presented in \figref{fig:synthetic_benchmark_pred_box}, suggesting that collecting a minimum amount of data-points with complete views is enough for our model to capture the joint relationship among views.
The empirical bound on this minimum level of data-points with all available views amounts to $25\%$.
In fact, in our synthetic tests, training on scenarios with completeness level above this bound does not seem to improve significantly the testing results.
This condition may be explained by the high collinearity between features due to the linear mappings used to generate the multi-view data.
The same bound may be noticed also in our showcase experiment (\S~\ref{sec:proof_of_concept}) where we jointly modeled MRI and FDG-PET brain images.
This results suggest that our model can reach its highest prediction power also when data collection resources are scarce,
such as in studies were the acquisition of complete observations is hampered by economical reasons or subject dropout.
% This condition may be due to overtraining, where collecting more data may not necessary to improve the model performance \citep{Bilbao2017}.
% We think that this fact can have a positive impact in studies were the acquisition of complete observations is hampered by economical or ethical reasons.

%\textcolor{blue}{
%\refnum{\#23}
	As a secondary result, we report the positive performance of knn (k = 5) in synthetic scenarios, especially in low snr cases, and on real data experiment, were it is most of the time superior to the DAE.
	This finding is corroborated by \cite{Platias2020} were knn is found to be superior to methods based on autoencoders.
%}

%\textcolor{blue}{
%\refnum{\#24, \#25}
	The experimental results on real medical imaging datasets (\tabref{tab:features}, \tabref{tab:classifier}) show on the horizontal axes the clear improvement of our MT-MCVAE method with respect to the MCVAE, that inspired our work,
	given the very same training and testing conditions for both of models.
	The features and diagnosis prediction clearly improves when using our method, that allows to not discard observations with missing views in the training phase.
	On the same tables, when read on the vertical axes, we note that models trained and tested on the same single dataset (STIB cases) tend to be more accurate than those trained on multiple other datasets (STEB cases).
	This is an expected result pointing to the issue of ``domain shift'', i.e. when observations coming from different datasets are not identically distributed,
	leading to generally high ``within task'' accuracy, and low generalization ability in the ``between task'' setting.
	We want to emphasize that we mitigated this problem with a data harmonization step based on ComBat \citep{combat},
	one of the state-of-the-art normalization method in biomedical applications \citep{Fortin2017, Fortin2018, Orlhac2020}.
	For this reason, we believe that the domain shift has a marginal impact for the application proposed in our study,
	and that those differences on the vertical axes are most likely due to the large variety of number of observations, available views, and differences in stratification by diseases in the datasets (\cf \tabref{table:datasets}, \tabref{tab:dx_stratification}).
%}

In feature prediction experiments (\tabref{tab:features_mtl}) we showed that MT-MCVAE models trained jointly on multiple neuroimaging datasets (ADNI, MIRIAD, OASIS-3, Geneva cohort)
perform generally better than the ones trained on a single dataset.
We suspect that there are two reasons explaining these results.
The first is that modeling simultaneously multiple datasets with our method brings more variability and information at play, making the generalization to unseen data less prone to prediction errors.
The second reason maybe that every decoder, associated to its specific view, acts, through the shared latent space, as a regularizer for all the other decoders.

In experiments where we seek to classify subjects to predict their cognitive status (\tabref{tab:classifier_mtl}),
the MT-MCVAE generalizes better to new unseen datasets when trained jointly on multiple datasets (MTL cases) with respect to cases where the training happens on a single dataset.
We notice that the best results happen in cases where testing data and training data come from the same dataset (ST cases), that is when the testing dataset is not anymore unseen to our model.
This is a different result than the analogous one in the feature prediction experiments, and we argue that the reason may be due to the lack of the regularization mechanism induced by having concurring decoders.
Indeed, the MT-MCVAE classifier is composed by a single decoder only, which can become highly specialized in decoding testing data coming from the same dataset of the training data.

In our non linear experiments we did not capture any improvement by using deep architectures with respect to simple linear mappings,
in both feature prediction \tabrefp{tab:features_mtl_nl} and classification tasks \tabrefp{tab:classifier_mtl_nl} on real neuroimaging datasets.
These results are in line with our previous work \citep{Antelmi2019}, were we benchmarked other auto-encoding based methods on observations coming from the ADNI dataset.
We suspect that this result is due to the general high heterogeneity and relatively small sample size of typical neuroimaging data.
%\textcolor{blue}{
%\refnum{\#11}
	Our results on the classification task in multi-view and multi-dataset problem also showed that our approach is equivalent to the EmbraceNet \citep{embracenet} recently proposed in the literature (\S~\ref{ssec:classifier}).
	While this finding indicates the ability of MT-MCVAE to provide results compatible with the state of the art in MTL classification problems,
	we note that the architecture of our framework enables a much larger set of applications than the one tackled by the EmbraceNet,
	such as cross-modality reconstruction and cross-dataset dimensionality reduction.
	%We also report no significant differences between the EmbraceNet and our MT-MCVAE \tabrefp{tab:classifier_mtl_nl},
	%although we note that is only with the latter that we have a framework adaptable to both classification and modality-to-modality prediction tasks.
%}

In our work we have thoroughly investigated architectures with a one-to-one correspondence between encoding and decoding views.
This makes our model part of the family of the auto-encoders, where the model acts as identity transformation between the input and the output.
Other architectures are nevertheless possible, such as the classifier described in \S~\ref{ssec:classifier}
In general, there may be an $m$-to-$n$ relationship, with partially overlapping views among $m$ input views and $n$ output views.
Investigating the properties of all the possible architectures is beyond the scope of this work.

%\textcolor{blue}{
%\refnum{\#24}
	As final remark, we want to stress that our model is based on the assumption of independent and identical distributed observations.
	This assumption may be limiting in healthcare datasets, such as the ones used in this work.
	In our work we mitigated these biases by harmonizing the datasets before applying our model, and we leave the extension and development of a bias-transparent multi-view models to future works.
%}
